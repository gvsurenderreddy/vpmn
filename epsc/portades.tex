%%%%%%%%%%%%%%%%%%%%%%%%%%%%%%%%%%%%%%%%%%%%%%%%%%%%%%%%%%%%%%%%%%%%%%%%%%%%%
%% PORTADA
%%%%%%%%%%%%%%%%%%%%%%%%%%%%%%%%%%%%%%%%%%%%%%%%%%%%%%%%%%%%%%%%%%%%%%%%%%%%%
%% Portada generada automàticament a partir del fitxer de dades
\portada

%% RESUM
%%%%%%%%%%%%%%%%%%%%%%%%%%%%%%%%%%%%%%%%%%%%%%%%%%%%%%%%%%%%%%%%%%%%%%%%%%%%%
% NOTA: la longitud passada com a parametre d'entrada 
%       s'ha d'ajustar ``a ull'' fins que el requadre del resum ocupi tota la pagina 
\begin{resum}{3.7cm}
% El resum ha de tenir una extensió mínima de 1500 caràcters sense
% espais i una extensió màxima de 3000.
Aquest projecte ens apropa als conceptes com les xarxes privades virtuals, els serveis de seguretat (xifrar el tràfic, validar la integritat, autenticar els extrems, evitar el repudi i evitar la repetició) i les aplicacions distribuïdes (\emph{peer to peer}).

Avui en dia, la funció de permetre la unió de diferents ordenadors o xarxes locals en una nova xarxa virtual, pot ser aprofitada per poder abstraure's de les barreres imposades per IPv4 com la limitació d'adreçament públic. El fet d'afegir la component de privacitat permet forçar un entorn segur, de confiança i independent del que puguin aportar les aplicacions. El conjunt de la creació de xarxes virtuals amb la component de privacitat permet la creació de xarxes privades virtuals també anomenades VPNs.

L'objectiu d'aquest projecte és dissenyar i implementar una aplicació capaç de crear xarxes privades virtuals que no depenguin de cap servidor central, sense que això comprometi la privacitat ni l'autenticació dels integrants de la xarxa. L'aplicació ha de ser capaç de superar els routers NAT (que tradueixen les adreces IP permetent compartir una adreça pública entre diferents ordenadors) per tal d'establir connexions bidireccionals directament amb els veïns de la xarxa virtual, proporcionant així una baixa latència.

En fer els tests inicials es va trobar un problema relacionat amb la implementació de la llibreria OpenSSL del protocol segur utilitzat. Aquest error es presenta malgrat que aparentment l'aplicació fa un bon ús d'aquesta llibreria. Aquest problema ha consumit molt temps de dedicació del projecte sense poder ser solucionat.

Com a resultats dels tests de l'aplicació creada en comparació amb les de les altres aplicacions existents: aquesta realitza una inicialització breu, te una latència baixa juntament amb una desviació estàndard molt baixa i permet taxes de transferència altes en TCP i baixes en UDP.

Aquest document comença amb una introducció a les xarxes privades virtuals i al projecte.
Seguidament, en el primer capítol s'exposa la descripció i la comparativa de les tecnologies de xarxes privades virtuals existents.
En el segon s'explica el funcionament, el disseny i l'arquitectura de l'aplicació creada.
En el tercer es presenten els resultats de les proves realitzades amb l'aplicació creada.
I finalment hi ha les conclusions, la bibliografia i el glossari.
\end{resum}

%% OVERVIEW
%%%%%%%%%%%%%%%%%%%%%%%%%%%%%%%%%%%%%%%%%%%%%%%%%%%%%%%%%%%%%%%%%%%%%%%%%%%%%
% NOTA: la longitud passada com a parametre d'entrada 
%       s'ha d'ajustar ``a ull'' fins que el requadre del resum ocupi tota la pagina 
\selectlanguage{english}
\begin{overview}{4.7cm}
This project is close to the concepts like the virtual private networks, the security services (encrypt the traffic, validate the integration, authenticate both ends, avoid the repudiation and avoid repetitions) and the distributed applications (like peer to peer).

Nowadays, to allow to join different computers or local networks into a new virtual network could be used to escape from the current barriers imposed by IPv4 limitations, like the availability of public addresses. It also could be added an additional component to this like the privacy of the communications, it could provide a safe and confident environment independent from each one the user applications could supply. Both components as a whole allow to create virtual private networks abbreviated as VPNs.

The main target of this project is to design and implement an application that permits to create virtual private networks which did not need to any central server without any compromise in the privacy nor the confidence in the authentication of the virtual network members.
The application needs to be able to reach directly the other members of the virtual private network across NAT routers (network address translator routers, to share a single IP with various computers) allowing a low latency.

During the initial tests a problem had appeared related to the implementation of the security protocol used provided by OpenSSL library. This error appears even the usage of this library seams correct. A lot of time had been expended with this problem without solving it.

The results of the tests of the created application compared with the other existing applications are: small initialization, low latency together with a few standard deviation, high data rates with TCP and low ones in UDP.

This document starts with an introduction to the virtual private networks and to this work.
The first chapter describes and compares the existing virtual private network technologies.
The second one explains how the created application works, how is designed and its internal architecture.
The third one present the results of the test done with the created application.
Finally there are the conclusions, the bibliography and a glossary.
\end{overview}
\selectlanguage{catalan}

%% DEDICATORIA (opcional)
%%%%%%%%%%%%%%%%%%%%%%%%%%%%%%%%%%%%%%%%%%%%%%%%%%%%%%%%%%%%%%%%%%%%%%%%%%%%%
%\begin{dedicatoria}
%Escriure aquí opcionalment la dedicatòria.
%\end{dedicatoria}

%% INDEX de continguts
%%%%%%%%%%%%%%%%%%%%%%%%%%%%%%%%%%%%%%%%%%%%%%%%%%%%%%%%%%%%%%%%%%%%%%%%%%%%%
\thispagestyle{empty}
\pdfbookmark[1]{Índexs}{indexcontingut}
\pdfbookmark[2]{Índex}{indexcontingut}  
\tableofcontents
%\cleardoublepage

%% INDEX de figures (opcional, comentar les 3 linies si no es desitja)
%%%%%%%%%%%%%%%%%%%%%%%%%%%%%%%%%%%%%%%%%%%%%%%%%%%%%%%%%%%%%%%%%%%%%%%%%%%%%
\vspace{5em}
%\thispagestyle{empty}
\pdfbookmark[2]{Índex de figures}{indexfigures} 
\listoffigures
%\cleardoublepage

%% INDEX de taules (opcional, comentar les 3 linies si no es desitja)
%%%%%%%%%%%%%%%%%%%%%%%%%%%%%%%%%%%%%%%%%%%%%%%%%%%%%%%%%%%%%%%%%%%%%%%%%%%%%
\vspace{2em}
%\thispagestyle{empty}
\pdfbookmark[2]{Índex de taules}{indextaules} 
\listoftables
\cleardoublepage

%%%%%%%%%%%%%%%%%%%%%%%%%%%%%%%%%%%%%%%%%%%%%%%%%%%%%%%%%%%%%%%%%%%%%%%%%%
%%%%%%                    INTRODUCCIÓ                          %%%%%%%%%%%
%%%%%%%%%%%%%%%%%%%%%%%%%%%%%%%%%%%%%%%%%%%%%%%%%%%%%%%%%%%%%%%%%%%%%%%%%%
%% NOTA: El text passat com a parametre d'entrada 
%%       és ''introducció'' amb l'idioma en que es redacti el projecte
\pagestyle{fancy} 

\begin{intro}{Introducció}
Internet actualment depèn extensivament de \index{IP}IP\footnote{Internet Protocol (versió 4 o 6)}v4 i veient que a la IPv6 encara li queda camí per convèncer per la implementació global. La funció de permetre la unió de diferents ordenadors o \index{LAN}LAN\footnote{Local Area Network}s en una nova xarxa virtual, pot ser aprofitada per poder abstraure's de les barreres que imposa la limitació d'adreçament públic de IPv4. La creació d'aquestes xarxes virtuals és molt útil, però s'ha de tenir en compte que qualsevol que tingui accés a aquesta xarxa serà com si estigués connectat directament amb les LANs i ordenadors que en formen part. És per això que casi tot programari que ofereix la creació d'aquestes xarxes virtuals, també ofereix la possibilitat d'afegir una capa de seguretat, que inclou la autenticació i també pot incloure l'encriptació del canal de dades. Aquesta solució es denomina \index{VPN}VPN\footnote{Virtual Private Network} i es la més utilitzada en tots els àmbits, casi tota xarxa virtual utilitzada és una VPN.
També es una realitat que la majoria d'aplicacions ja disposen d'una capa de seguretat pel transport de les pròpies dades, que faria poc útil haver de tornar a encriptar al passar per les VPNs.

La idea és la de desenvolupar una aplicació capaç de crear xarxes completament mallades a nivell de aplicació.

\end{intro}

\pagestyle{fancy} 

