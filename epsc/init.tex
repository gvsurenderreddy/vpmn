
\documentclass[a4paper,12pt,catalan,twoside,final]{epsc/epsc_tfc_pfc}
%% a4paper: mida paper. No tocar
%% 12pt: mida de la font. No tocar

%%  - OPCIONS A CONFIGURAR:
%%     - Estat del document: final o draft
%%       NOTA: Draft no inserta les figures i indica quan el text sobrepassa els marges.

%%  - IDIOMES QUE S'USARAN EN EL DOCUMENT: catalan, spanish, english, french...
%%    NOTA: per canviar d'idioma al mig del document usar:
%%          \selectlanguage{nom_idioma}
\usepackage[catalan,english]{babel}

%%%%%%%%%%%%%%%%%%%%%%%%%%%%%%%%%%%%%%%%%%%%%%%%%%%%%%%%%%%%%%%%%%%%%%%%%%%%%


%%% PAQUETS LATEX RECOMANABLES A UTILITZAR
%%%%%%%%%%%%%%%%%%%%%%%%%%%%%%%%%%%%%%%%%%%%%%%%%%%%%%%%%%%%%%%%%%%%%%%%%%%%%
%%% NOTA: es possible que algunes distribuicions Linux o Windows 
%%%       no portin aquests paquets instal·lats per defecte.

%% El paquet isolatin1 és extramadament útil. 
%% Permet escriure els accents directament amb l'editor de texte
%% sense haver de fer coses com per exemple: introducci\'o
%\usepackage{isolatin1} 
\usepackage{ucs}
\usepackage[utf8x]{inputenc}

%% Símbols matemàtics de la American Mathematical Society
\usepackage{amssymb,amsmath, amsfonts}  

%% El paquet array proporciona eines molt útils a l'hora de fer 
%% equacions amb matrius
%\usepackage{array}             

%% Permet fer taules fusionant cel·les de files consecutives
\usepackage{multirow}          

%% Permet canviar els colors del document
%\usepackage{color,colortbl}

%% Permet incloure pdfs
\usepackage{pdfpages}

% ifpdf package
\RequirePackage{ifpdf}

% hyperrefs
\RequirePackage{url}
\ifpdf
  \RequirePackage[pdftex,pdfpagelayout=TwoColumnRight]{hyperref}
\else
  \RequirePackage[dvips,pdfpagelayout=TwoColumnRight]{hyperref}\fi
\AtBeginDocument{%
    \hypersetup{%
      breaklinks,
      baseurl       = http://,
      pdfborder     = 0 0 0,
      pdfpagemode   = None,
      pdfpagelayout = TwoColumnRight,
      pdfstartview  = XYZ,
      pdfstartpage  = 1,
      bookmarksnumbered = true,
      pdfauthor     = \authordoc,%
      pdftitle      = \titledocE,%
      pdfsubject    = \titledoc,%
      pdfkeywords   = \titledoc~\titledocE,
      pdfcreator    = \LaTeX{},%
      pdfproducer   = \LaTeX}}
\urlstyle{tt}
