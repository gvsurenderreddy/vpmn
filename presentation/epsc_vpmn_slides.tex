\documentclass{beamer}
\mode<presentation>
{
  %\usetheme{Madrid}
  \setbeamercovered{transparent}
  \usefonttheme[onlylarge]{structurebold}
  \setbeamerfont*{frametitle}{size=\normalsize,series=\bfseries}
  \setbeamerfont*{date}{size=\tiny,series=\bfseries}
  \useinnertheme{rounded}
  \useoutertheme{infolines}
  \usecolortheme{whale}%dolphin}
  \usecolortheme{orchid}
  \setbeamertemplate{sidebar right}{
    \vfill
    \llap{
      \usebeamertemplate***{navigation symbols}
      \hskip0.1cm
      \insertlogo
      \hskip0.1cm
    }
    \vskip2pt
  }
  \setbeamertemplate{navigation symbols}{\hbox{
    \hbox{\insertframenavigationsymbol}
    \hbox{\insertsubsectionnavigationsymbol}
    \hbox{\insertsectionnavigationsymbol}
  }}
}

\usepackage[catalan]{babel}
% or whatever

\usepackage[utf8]{inputenc}
% or whatever

\usepackage{times}
\usepackage[T1]{fontenc}
% Or whatever. Note that the encoding and the font should match. If T1
% does not look nice, try deleting the line with the fontenc.

\title{Xarxa Mesh Privada Virtual} % (optional, use only with long paper titles)

\subtitle{Projecte de Fi de Carrera} % (optional)

\author{Pau~Rodriguez-Estivill}

\institute[UPC EPSC] % (optional, but mostly needed)
{
  Escola Polit\`ecnica Superior de Castelldefels\\
  Universitat Polit\`ecnica de Catalunya
}
% - Use the \inst command only if there are several affiliations.
% - Keep it simple, no one is interested in your street address.

\date{ % (optional)
\today
}

\subject{UPC EPSC PFC}
% This is only inserted into the PDF information catalog. Can be left
% out. 

% If you have a file called "university-logo-filename.xxx", where xxx
% is a graphic format that can be processed by latex or pdflatex,
% resp., then you can add a logo as follows:

% \pgfdeclareimage[height=0.5cm]{university-logo}{university-logo-filename}
% \logo{\pgfuseimage{university-logo}}

\pgfdeclareimage[height=0.5cm]{epsc-logo}{../epsc/logo_epsc}
\logo{\pgfuseimage{epsc-logo}}


\AtBeginSection[]
{
  \begin{frame}<beamer>{Índex}
    \tableofcontents[currentsection]
  \end{frame}
}

\AtBeginSubsection[]
{
  \begin{frame}<beamer>{Índex}
    \tableofcontents[currentsection,currentsubsection]
  \end{frame}
}

% If you wish to uncover everything in a step-wise fashion, uncomment
% the following command: 

%\beamerdefaultoverlayspecification{<+->}

\begin{document}
\pdfbookmark[2]{Benvinguda}{benvinguda}
\begin{frame}
%  \titlepage
  \vbox{}
  \vfill
  \begin{centering}
    \begin{beamercolorbox}[sep=8pt,center,colsep=-4bp,rounded=true]{title}
      \usebeamerfont{title}\inserttitle\par%
      \ifx\insertsubtitle\@empty%
      \else%
        \vskip0.25em%
        {\usebeamerfont{subtitle}\usebeamercolor[fg]{subtitle}\insertsubtitle\par}%
      \fi%     
    \end{beamercolorbox}%
    \vskip1em\par
    \begin{beamercolorbox}[sep=8pt,center,colsep=-4bp,rounded=true]{author}
      \usebeamerfont{author}\insertauthor
    \end{beamercolorbox}
    \begin{beamercolorbox}[sep=8pt,center,colsep=-4bp,rounded=true]{institute}
      \usebeamerfont{institute}
      \tiny\textbf{Director:} Jos\'e Manuel Y\'ufera-G\'omez\\~\\
      \textit{Enginyeria de Telecomunicaci\'o (segon cicle)}\\~\\
      \usebeamerfont{institute}\insertinstitute
    \end{beamercolorbox}
    \begin{beamercolorbox}[sep=8pt,center,colsep=-4bp,rounded=true]{date}
      \usebeamerfont{date}\insertdate
    \end{beamercolorbox}
  \end{centering}
  \vfill
\end{frame}

\begin{frame}{Índex}
  \tableofcontents
  % You might wish to add the option [pausesections]
\end{frame}

% However, the talk length of between 15min and 45min and
% the theme suggest that you stick to the following rules:

% - Exactly two or three sections (other than the summary).
% - At *most* three subsections per section.
% - Talk about 30s to 2min per frame. So there should be between about
%   15 and 30 frames, all told.

\section{Introducció}
    \begin{frame}{VPNs}
        \begin{center}
        \includegraphics[height=6em]{images/vpn}
        \end{center}
    \end{frame}
    \begin{frame}{Objectiu}
        \begin{center}
        \includegraphics[height=10em]{images/vpn-fullymeshed}
        \end{center}
    \end{frame}

\chapter{Estudi de la situació actual}
\section{Protocols i programaris existents}
Els protocols de seguretat poden proveir alguns dels serveis de seguretat aquí explicats:
\begin{itemize}
\item \textbf{Xifrar el tràfic} de manera que no pugi ser llegit per ningú més que pels destinataris originals.
\item \textbf{Validar la integritat} per assegurar que el tràfic no ha estat modificat durant el seu recorregut.
\item \textbf{Autenticar els extrems} per assegurar que el tràfic prové d'un extrem de confiança.
\item \textbf{Evitar el repudi} de l'altre extrem per tal de que no pugui negar haver enviat l'informació.
\item \textbf{Evitar la repetició} malintencionada de paquets (l'anti-repetició).
\end{itemize}

A continuació s'analitzaran alguns protocols i programaris existents.
\subsection{IPsec}
\keyword{IPsec}{Internet Protocol security} és d'ús opcional en IPv4 i serà obligatori en IPv6. IPsec va ser creat per proporcionar seguretat en dos possibles modes. El mode de transport (extrem a extrem), en el que els ordenadors dels extrems finals realitzen el processat de seguretat del tràfic de paquets. El mode túnel (porta a porta) en el que la seguretat del tràfic de paquets és proporcionada a vàries màquines (inclús a tota la LAN) per un únic node.

IPsec va ser introduït per proporcionar serveis de seguretat tals com: xifrar el tràfic, validar la integritat, autenticar als extrems i l'anti-repetició. L'ús principal d'IPsec és el de crear VPNs en qualsevol dels dos modes, però les implicacions de seguretat són bastant diferents entre els dos modes d'operació.

La seguretat de comunicacions extrem a extrem a escala Internet és va desenvolupar més tard i utilitza la infraestructura de clau pública universal \keyword{DNSSEC}{Domain Name System Security Extensions}.

La majoria d'implementacions d'aquest protocol tenen problemes de compatibilitat amb les altres implementacions del mateix protocol, degut a que cada fabricant interpreta el \rfckeyword{} i s'adapta el protocol a la seva mida. L'ús de DNSSSEC per a l'intercanvi de claus no es troba en gaires implementacions.

Així doncs depenent del nivell en el que actuï l'IPsec, aquest estarà treballant en mode transport o mode túnel.
\subsubsection{Mode transport}
En mode transport, només es xifra i/o autentica el \emph{payload} (la carga útil) del paquet IP. Per tant al no tocar les capçaleres IP no afecta a l'enrutament, però si s'utilitza l'\keyword{AH}{Authentication Header} les direccions IP no poden ser traduïdes (com fa per exemple un NAT) ja que això faria que el valor del \emph{hash} (resum) no coincidís. Aquest mode s'utilitza en comunicacions ordinador a ordinador.

Per tal de travessar els NATs s'ha definit en un RFC com a mecanisme d'encapsulació de missatges IPsec anomanat \keyword{NAT-T}{NAT Traversal in the IKE}.
\subsubsection{Mode túnel}
En el mode túnel, es xifra i/o autentica tot el paquet IP sencer (capçalera inclusiva). Aquesta informació s'encapsula dins del \emph{payload} d'un nou paquet IP per tal de que pugui ser enrutat. Aquest mode s'utilitza en comunicacions xarxa a xarxa, per exemple per crear VPNs a través d'Internet.

\subsection{OpenVPN}
OpenVPN és un programa de VPN a nivell d'aplicació, que fa ús del \emph{driver} genèric i multiplataforma TUN/TAP. Mijançant aquest \emph{driver} l'aplicació pot crear VPNs en dos modes:
\begin{itemize}
\item \textbf{Mode pont Ethernet}
Simula una interfície de xarxa Ethernet i crea una VPN que treballa amb trames de capa 2 del \keywords{OSI}{model OSI}{Open Systems Interconnection - Basic Reference Model}.
\item \textbf{Mode túnel IP}
Simula una interfície de xarxa punt-a-punt i crea una VPN que treballa amb paquets de capa 3 del model OSI.
\end{itemize}

OpenVPN en mode pont Ethernet sempre utilitza una única interfície de xarxa en cada màquina per tota la VPN. Peró en el mode túnel IP pot treballar en diferents topologies internes:
\begin{itemize}
\item \textbf{p2p}: Utilitza una interfície de xarxa punt-a-punt al servidor per a cada client connectat, i el sistema operatiu del servidor s'encarrega de l'enrutat. Els clients només veuen el servidor i han d'utilitzar-lo com a \emph{router} per accedir a la resta de clients de la VPN. Aquest mode no és compatible amb les màquines que utilitzin Microsoft Windows.
\item \textbf{net30}: Utilitza una interfície de xarxa al servidor per a cada client connectat. Tant els clients com el servidor en les seves interfícies, utilitzen subxarxes /30 i per tant el sistema operatiu del servidor també s'encarrega de l'enrutat. Els clients segueixen veient només al servidor i han d'utilitzar-lo com a \emph{router} per accedir a la resta de clients de la VPN. Aquest mode és compatible amb les màquines que utilitzin Microsoft Windows.
\item \textbf{subnet}: Utilitza una única interfície de xarxa en cada màquina per tota la VPN, com en el mode Ethernet. Per tant, és l'aplicació qui s'encarrega de l'enrutat de la VPN.
\end{itemize}

L'aplicació te diferents modes de funcionament:
\begin{itemize}
\item \textbf{Mode punt-a-punt}: Crea una VPN amb només 2 extrems.
\item \textbf{Mode servidor}: Permet que se li connectin varis clients.
\item \textbf{Mode client}: Permet connectar-se al servidor.
\end{itemize}
Per tant podem dir que és tracta d'una aplicació pensada per VPNs centralitzades, en la figura \ref{F:vpn-centralized} es mostra un diagrama d'aquesta topologia.
\begin{figure}[htb]
\centering
\includegraphics[height=0.5\textwidth]{images/vpn-centralized}
\caption{Topologia d'una VPN centralitzada}
\label{F:vpn-centralized}
\end{figure}

A nivell de seguretat utilitza la llibreria Open\keyword{SSL}{Secure Socket Layer (la versió 3 en convertir-se en estàndard es va anomenar TLS)} tant per \keyword{TCP}{Transmission Control Protocol} com per \keyword{UDP}{User Datagram Protocol}, en TCP utilitza l'estàndard \keyword{TLS}{Transport Layer Security} i en UDP utilitza un protocol propi basat en TLS. Permet l'autenticació per claus secretes compartides prèviament, per certificats x509 i per usuari-contrasenya.

\subsection{TincVPN}
TincVPN és un programa de VPNs a nivell d'aplicació, que fa ús del \emph{driver} genèric i multiplataforma TUN/TAP. Mitjançant aquest \emph{driver} l'aplicació pot crear VPNs en \textbf{mode pont Ethernet} o en \textbf{mode túnel IP}, ja explicats anteriorment. L'aplicació crea una VPN mallada (figura \ref{F:vpn-meshed}) connectant-se a una llista de \emph{peers}, i la resta de \emph{peers} de la VPN hauran de ser enrutats a través d'altres \emph{peers} per arribar als nodes que no tinguin connexió directe.
\begin{figure}[htb]
\centering
\includegraphics[height=0.5\textwidth]{images/vpn-meshed}
\caption{Topologia d'una VPN mallada}
\label{F:vpn-meshed}
\end{figure}

Utilitza una única interfície de xarxa en cada màquina per tota la VPN, com en la topologia interna \emph{subnet} de OpenVPN. Per tant, és cada \emph{peer} qui s'encarrega del \emph{bridging} (en mode pont Ethernet) o de l'enrutat (en mode túnel IP) de la VPN a nivell d'aplicació.

A nivell de seguretat utilitza un protocol propi tant en TCP com en UDP. Peter Gutmann va trobar nombroses errades de disseny en la seguretat del protocol de la versió 1 d'aquest programa que encara no s'han resolt, a l'annex \refannexmail{} es troba l'e-mail on exposa el seu anàlisi.

\subsection{Wippien}
Wippien és un programa de missatgeria instantània amb funcionalitats de creació de VPNs a nivell d'aplicació. Utilitza \keyword{XMPP}{Extensible Messaging and Presence Protocol} com a missatgeria instantània. Una vegada connectat al servidor XMPP, l'aplicació mostra a l'usuari la seva llista de contactes per tal que esculli amb quins contactes vol crear la VPN. Per crear-la negocia els paràmetres de la VPN (com la IP pública, el port i el protocol) a través d'XMPP. També utilitza HTTP cap a un servidor anomenat \emph{mediator}, aquest fa la funció de \keyword{DHCP}{Dynamic Host Configuration Protocol} assignant una IP única per sessió a cada \emph{peer} de la VPN, d'aquesta manera es garanteix que no hi hagi conflictes d'IPs.
\begin{figure}[htb]
\centering
\includegraphics[height=0.5\textwidth]{images/vpn-externalized}
\caption{Topologia d'una VPN amb autenticació externa}
\label{F:vpn-externalized}
\end{figure}
Per tant podem dir que Wippien crea VPNs mallades i completament connectades per cada anella de confiança, amb autenticació (i assignació de les IPs) per part d'un servidor central extern (veure la figura \ref{F:vpn-externalized}).

Utilitza una única interfície de xarxa en cada màquina per tota la VPN, com en la topologia interna \emph{subnet} de OpenVPN. Per tant, és cada \emph{peer} qui s'encarrega de l'enrutat de la VPN a nivell d'aplicació. Totes les IPs de les VPNs que utilitzin el seu servidor \emph{mediator} pertanyen a la xarxa 5.0.0.0/8.

A nivell de seguretat utilitza un protocol propi amb AES128, mitjançant un component privatiu anomenat \emph{wodVPN ActiveX}, aquest component és \emph{NAT friendly} (està preparat per travessar els routers NAT) i suporta tant TCP com UDP.

\subsection{Hamachi}
Hamachi és un programa privatiu de VPNs a nivell d'aplicació que suporta tant túnels IP com túnels \keyword{IPX}{Internetwork Packet Exchange}.
El programa utilitza una connexió a un servidor central com a canal de control, en arrancar efectua el procés d'autenticació amb aquest servidor i efectua el descobriment del canal del client per assegurar la connectivitat en cas d'existir routers NAT (per tant és una aplicació \emph{NAT friendly}). Finalment el servidor entrega al client la seva llista de VPNs juntament amb la llista dels membres que s'actualitza constantment cada cop que aquests entren o surten de l'aplicació. Tal com s'ha dit el servidor ajuda: en el procés de superar els routers NAT, en el procés d'autenticació i en el descobriment de \emph{peers} (veure la figura \ref{F:vpn-externalized}).

Utilitza una única interfície de xarxa en cada màquina per tota la VPN. Per tant, és cada \emph{peer} qui s'encarrega de l'enrutat de la VPN a nivell d'aplicació. Totes les IPs de les VPNs pertanyen a la xarxa 5.0.0.0/8.

A nivell de seguretat la VPN utilitza un protocol propi sobre UDP.

\subsection{ELA VPN}
\keyword{ELA}{Everywhere Local Area network} és un programa de VPN a nivell d'aplicació.
Utilitza una topologia de nodes mallada (veure la figura \ref{F:vpn-meshed}) connectada tant amb un protocol propi tant amb TCP com amb UDP.
D'aquesta tecnologia no s'ha trobat cap implementació, el \emph{paper} descriptiu es troba adjunt a l'annex \refannexpapers.

\section{Comparativa}
Per acabar, la taula següent (\ref{T:statecomp}) fa un resum de les tecnologies analitzades en aquest capítol.
\begin{table}[htb]
\begin{center}
\begin{tabular}{|l|c|c|c|c|}
\hline
Nom & Lliure & Topologia & Protocol & Seguretat \\ \hline \hline
\bf IPsec & Sí & Descentralitzada & IP & Estàndard \\ \hline
\bf OpenVPN & Sí & Centralitzada & TCP/UDP & TLS/Propi \\ \hline
\bf TincVPN & Sí & Mallada & TCP/UDP & Propi \\ \hline
\bf Wippien & No & Mallada amb Servidor & TCP/UDP & Propi \\ \hline
\bf Hamachi & No & Mallada amb Servidor & UDP & Propi \\ \hline
\bf ELA VPN & No & Mallada Jeràrquica & TCP/UDP & Propi \\ \hline
\end{tabular}
\end{center}
\begin{center}
\caption{Comparativa dels protocols i programaris existents}
\label{T:statecomp}
\end{center}
\end{table}

\section{Programació}
\subsection{Objectiu de l'aplicació}
    \begin{frame}{Funcionament}
        \begin{center}
        \includegraphics[height=10em]{images/vpn-fullymeshed}
        \end{center}
        \begin{itemize}
\item P
        \end{itemize}
    \end{frame}
\subsection{Disseny del protocol}
    \begin{frame}{Seguretat}
        \begin{itemize}
\item DTLS
        \end{itemize}
        \begin{block}{NameConstraints (x509v3)}
        nameConstraints=permitted;IP:192.168.0.0/255.255.0.0
        \end{block}
    \end{frame}
    \begin{frame}{Fluxe}
        \begin{center}
        \includegraphics[height=12em]{images/dia-pktflow}
        \end{center}
        \begin{itemize}
\item Paquets Identification (ID)
\item Paquets Identification Acknowledgment (ID ACK)
\item Paquets Keep Alive (KA)
\item Paquets Internet Protocol (IPv4)
        \end{itemize}

    \end{frame}
    \begin{frame}{Paquets Identification}
        \begin{center}
\scriptsize
\begin{tabular}{|c|p{0.0625\linewidth}|p{0.0625\linewidth}|p{0.125\linewidth}|p{0.25\linewidth}c|}
\hline
bits & \centering 0--3 & \centering 4--7 & \centering 8--15 & \centering 16--31 & \\ \hline \hline
0 & \centering 0000 & \centering 0001 & \centering 0x00 & \centering Total Lenght & \\ \hline
32 & \multicolumn{2}{|c|}{\# Networks} & \centering \# IP-Ports & \\ \cline{0-3} \noalign{\vskip 2pt} \hline
48 & \multicolumn{4}{|c}{Network IP} & \\ \hline
80 & \multicolumn{4}{|c}{Network Netmask} & \\ \hline
112 & \multicolumn{4}{|c}{\ldots} & \\ \hline
144 & \multicolumn{4}{|c}{\ldots} & \\ \cline{0-5} \noalign{\vskip 2pt} \cline{0-5}
=0 & \multicolumn{4}{|c}{Host IP} & \\ \hline
+32 & \multicolumn{3}{|c|}{UDP Port} & \\ \hline
+48 & \multicolumn{4}{|c}{\ldots} & \\ \hline
+80 & \multicolumn{3}{|c|}{\ldots} & \\ \cline{0-3}
\multicolumn{6}{c}{~} \\
\multicolumn{6}{c}{~} \\
\multicolumn{6}{c}{~} \\
\hline
bits & \centering 0--3 & \centering 4--7 & \centering 8--15 & \centering 16--31 & \\ \hline \hline
0 & \centering 0000 & \centering 0001 & \centering 0x01 & \centering 0x04 & \\ \hline
\end{tabular}
        \end{center}
    \end{frame}
    \begin{frame}{Paquets Keep Alive}
        \begin{center}
\scriptsize
\begin{tabular}{|c|p{0.0625\linewidth}|p{0.0625\linewidth}|p{0.125\linewidth}|p{0.25\linewidth}c|}
\hline
bits & \centering 0--3 & \centering 4--7 & \centering 8--15 & \centering 16--31 & \\ \hline \hline
0 & \centering 0000 & \centering 0001 & \centering 0x02 & \centering Total Lenght & \\ \hline
32 & \multicolumn{2}{|c|}{\# Peers} \\ \cline{0-2} \noalign{\vskip 2pt} \cline{0-3}
40 & \multicolumn{2}{|c|}{\# Networks} & \centering \# IP-Ports & \\ \hline
56 & \multicolumn{4}{|c}{Network IP} & \\ \hline
88 & \multicolumn{4}{|c}{Network Netmask} & \\ \hline
120 & \multicolumn{4}{|c}{\ldots} & \\ \hline
152 & \multicolumn{4}{|c}{\ldots} & \\ \hline
=0 & \multicolumn{4}{|c}{Host IP} & \\ \hline
+32 & \multicolumn{3}{|c|}{UDP Port} & \\ \hline
+48 & \multicolumn{4}{|c}{\ldots} & \\ \hline
+80 & \multicolumn{3}{|c|}{\ldots} & \\ \cline{0-3} \noalign{\vskip 2pt} \cline{0-3}
=0 & \multicolumn{2}{|c|}{\ldots} & \centering \ldots & \\ \hline
+16 & \multicolumn{4}{|c}{\ldots} & \\ \hline
+48 & \multicolumn{4}{|c}{\ldots} & \\ \hline
=0 & \multicolumn{4}{|c}{\ldots} & \\ \hline
+32 & \multicolumn{3}{|c|}{\ldots} & \\ \cline{0-3}
\end{tabular}
        \end{center}
    \end{frame}
\subsection{Arquitectura del programa}
    \begin{frame}{Processos}
        \begin{center}
        \includegraphics[height=12em]{images/dia-app}
        \end{center}
    \end{frame}
    \begin{frame}{UDP Server Part}
        \begin{center}
        \includegraphics[height=19em]{images/dia-udpsrv}
        \end{center}
    \end{frame}
    \begin{frame}{TUN Server Part}
        \begin{center}
        \includegraphics[height=18em]{images/dia-tunsrv}
        \end{center}
    \end{frame}
\subsection{Gestió de la VPN}
    \begin{frame}{Casos d'ús}
        \begin{center}
        \includegraphics[height=18em]{images/dia-case}
        \end{center}
    \end{frame}

\chapter{Tests i resultats}
\section{Metodologia}
En aquest capítol s'exposaran els resultats dels tests realitzats per comparar l'aplicació creada amb el programari existent.
En tots els programes s'han utilitzat certificats de 1024 bits i s'ha activat la compressió que permetés l'aplicació; el motiu d'activar la compressió és que la llibreria OpenSSL utilitzada per la creació del \keyword{VPMN}{Virtual Private Mesh Network} (el programa creat en aquest projecte) no permet desactivar la compressió en temps d'execució.
L'escenari dels tests consta d'una xarxa Gigabit Ethernet on hi estan connectats dos ordenadors:
\begin{enumerate}
\item Debian GNU/Linux (PC amd64)
\begin{itemize}
\item AMD Athlon 64 X2 Dual Core Processor BE-2400 (2.3GHz, 1024KB cache)
\item nForce Gigabit Ethernet CK804
\end{itemize}
\item Debian GNU/Linux (PC i686)
\begin{itemize}
\item Intel Pentium M Centrino Processor (1.70GHz, 2048KB cache)
\item Realtek Gigabit Ethernet RTL-8169
\end{itemize}
\end{enumerate}
%TODO:
%Afegir retards als tests
%Tests de Compressio
\subsection{Eficiència}
Per analitzar l'eficiència dels programes s'analitzen els bytes transmesos durant la inicialització, en el cas de la connexió directe es conten els bytes de les peticions \keyword{ARP}{Address Resolution Protocol} i en la resta de casos s'exclouen.
També s'analitzen els bytes transmesos durant l'enviament de paquets \keyword{ICMP}{Internet Control Message Protocol} de \emph{echo} (\emph{ping}) de 56 bytes (contant la capa IP són 84 bytes/paquet).
Els resultats es contaran en bytes sense comptar la capa Ethernet (14 bytes/trama).

En el moment d'analitzar els resultats s'ha de tenir en compte que TincVPN no intercanvia certificats durant la inicialització, en el seu lloc les claus públiques dels altres nodes s'han d'instal·lar al configurar el programari (per tenir una idea del que ocupen 2 certificats de 1024 bits: $830\cdot2=1660$ bytes).
També s'ha de tenir en compte que els diferents programes utilitzen diferents algoritmes de compressió. 

\subsection{Latència}
Per analitzar la latència dels diferentes programes es realitzen les mesures amb el programa \emph{ping}. S'han realitzat 100 mesures en intervals de 2 segons per donar temps als algoritmes del programari VPN.

\subsection{Taxa màxima de transferència}
Per analitzar la taxa màxima de transferència s'han realitzat els tests amb 2 programes diferents: netperf i MGEN.
El netperf s'ha utilitzat per realitzar tests en TCP i UDP, aquests tests es realitzen enviant grans blocs d'informació a traves dels protocols esmentats.
El MGEN s'ha utilitzat amb la configuració de la taula \ref{T:mgencfg} per tal d'enviar paquets a 737.28 Mbps durant 10 segons, i mesurant el nombre de paquets rebuts s'ha calculat la taxa rebuda.
\begin{table}[htb]
\begin{center}
\begin{minipage}[htb]{0.6\linewidth}
\footnotesize
\begin{verbatim}
TXBUFFER 1000
0.0 ON  1 UDP DST 10.0.0.1/5000 PERIODIC [90000.0 996]
10.0 OFF 1
\end{verbatim}
\end{minipage}
\caption{Configuració MGEN}
\label{T:mgencfg}
\end{center}
\end{table}

En la taula \ref{T:mgencfg}, la configuració del MGEN utilitzada, encomana enviar 90000 paquets UDP per segon amb un \emph{payload} UDP de 996 bytes, resultant així paquets IP de 1024 bytes. Per a calcular la taxa de recepció s'utilitza l'equació \ref{E:mgen}, tenint en compte que l'activitat del MGEN es realitza només durant 10 segons.

\begin{minipage}[htb]{\linewidth}
\begin{equation}\label{E:mgen}
r=P\cdot\frac{(996+28)\cdot8}{10}
\end{equation}
\centering
{\scriptsize
r: Taxa de recepció. 
P: Número de paquets rebuts. 
}\\
%Equació \ref{E:mgen}: Càlcul de la tassa de recepció del MGEN.
\vspace{1em}
\end{minipage}

En interpretar els resultats obtinguts s'ha de tenir en compte la presència de la compressió durant els tests del programari VPN. Aquesta pot afectar a la comparativa de la taxa de recepció si el tràfic generat té diferents ratis de compressió depenent dels algoritmes utilitzats en cada programari.

La llibreria OpenSSL només pot treballar amb un sol \emph{thread} per connexió, per tant la utilització de \emph{threads} en l'aplicació creada en aquest projecte (VPMN) només es veuria beneficiada en un context de vàries connexions.

%Proves amb varis nodes
%\subsection{Capacitat}
%\subsection{Càrrega}
%\subsection{Escalabilitat}

\section{Anàlisi de resultats}
Abans d'analitzar els resultats cal comentar els problemes trobats durant les proves del programa creat. 
Ja durant els primers tests d'aquesta aplicació es van notar comportaments estranys amb la recepció de tràfic xifrat i s'ha invertit molt de temps en intentar trobar l'arrel d'aquest problema. Per mirar de sol·lucionar el problema es va intentar canviar la part que treballa amb el xifratge amb diferents algoritmes i estructures. Finalment després de depurar molt el programa s'ha pogut trobar que el comportament anòmal ve donat per la llibreria OpenSSL, que sembla que falla al cap d'un temps aleatori malgrat aparentment l'aplicació fa un bon ús d'ella. Per tant sembla que la recent implementació de DTLS d'aquesta llibreria conté errors que afecten a l'ús que s'en fa en aquesta aplicació.
El poc ús que rep aquest protocol, i per tant la seva implementació en la OpenSSL, fa que sigui més difícil trobar documentat aquest possible error, però s'ha començat a buscar l'error dins del codi font de la llibreria.

Com a conseqüència dels problemes esmentats els tests d'aquesta aplicació, aquests s'han hagut de realitzat nombroses vegades abans de poder obtenir un resultat.

Per analitzar els resultats cal tenir en compte que tal com s'ha descrit l'escenari algunes aplicacions tenen un comportament diferent:
\begin{itemize}
\item Les aplicacions utilitzen algoritmes de compressió i xifratge diferents.
\item El OpenVPN no realitza tasques d'enrutament intern en el node client al haver-hi una instància servidor i una client, tant el TincVPN com el VPMN realitzen aquestes tasques en ambdós nodes.
\item El TincVPN no intercanvia certificats ni claus públiques; el seu protocol, tal com esta descrit en el capítol \ref{TincVPN-Sec}, no compleix tots els requisits per a ser un protocol segur.
\end{itemize}

En la taula \ref{T:efi} es compara la \textbf{eficiència} en el tràfic.
\begin{table}[htb]
\begin{center}
\begin{tabular}{|c|r|r|}
\multicolumn{1}{c}{} & \multicolumn{2}{|c|}{Mida (bytes)} \\ \hline
Aplicació & Inicialització & Ping \\ \hline \hline
\tt directe & 112 & 84 \\ \hline
OpenVPN & 11876 & 153 \\ \hline
TincVPN & 2569 & 136 \\ \hline
\bf VPMN & 4641 & 169 \\ \hline
\end{tabular}
\end{center}
\begin{center}
\caption{Eficiència}
\label{T:efi}
\end{center}
\end{table}
Es pot veure com el OpenVPN intercanvia notablement més informació durant la inicialització, més del doble del trafic generat pel VPMN.
El TincVPN es manté per sota de les altres aplicacions, però aquest no intercanvia certificats durant la inicialització: afegint un intercanvi de 2 certificats serien 4229 bytes i estaria a prop del VPMN però encara mantenint-se per sota; afegint els 2 certificats més, el de la CA de cada un, aquest hauria d'enviar 5889 bytes i es quedaria entre el OpenVPN i el VPMN. Per tant es pot dir que TincVPN envia poca informació d'inicialització gràcies a l'absència d'aquest intercanvi.
La negociació ARP de la connexió directe esta molt per sota de totes les aplicacions, per tant com ja es podia predir el contingut que necessiten transportar els paquets ARP es molt menor al de qualssevol negociació de les aplicacions VPN.

També es el TincVPN el que aconseguix la millor eficiència en la transferència dels paquets \emph{ping}. Les altres aplicacions resulten estar molt a prop d'aquest resultat però aquesta vegada OpenVPN millora l'eficiència al VPMN. En aquest test la mida original, mesurat en la connexió directe, no s'allunya molt dels resultats dels paquets xifrats.

En la taula \ref{T:lat} es compara la \textbf{latència} mesurant el \keyword{RTT}{Round-Trip Time}.
\begin{table}[htb]
\begin{center}
\begin{tabular}{|c|c|c|c|c|c|}
\multicolumn{2}{c}{} & \multicolumn{4}{|c|}{RTT (ms)} \\ \hline
Aplicació & Pèrdues & Mínim & Mitjana & Màxim & D.Estàndard \\ \hline \hline
\tt directe & 0\% & 0.047 & 0.050 & 0.074 & 0.006 \\ \hline
OpenVPN & 0\% & 0.199 & 0.479 & 5.333 & 1.080 \\ \hline
TincVPN & 0\% & 0.136 & 0.185 & 0.911 & 0.092 \\ \hline
\bf VPMN & 0\% & 0.228 & 0.310 & 0.415 & 0.034 \\ \hline
\end{tabular}
\end{center}
\begin{center}
\caption{Latència}
\label{T:lat}
\end{center}
\end{table}
La mínima latència és la del TincVPN i la màxima és la del OpenVPN. TincVPN es manté entremig però amb una desviació estàndard molt inferior a la dels altres dos programaris. La desviació estàndard major es la de OpenVPN, aquest també enregistra valors màxims molt elevats per estar en una xarxa local. Els resultats de la latència en connexió directe només donen una referencia de que l'escenari utilitza una xarxa local.

En la taula \ref{T:tax} es compara la \textbf{taxa màxima de transferència}.
\begin{table}[htb]
\begin{center}
\begin{tabular}{|c|c|c|r|}
\multicolumn{1}{c}{} & \multicolumn{3}{|c|}{Màxim (Mbps)} \\ \hline
Aplicació & TCP & UDP & MGEN \\ \hline \hline
\tt directe & 690.49 & 929.91 & 116.03 \\ \hline
OpenVPN & 185.01 & 500.37 & 78.52 \\ \hline
TincVPN & 133.26 & 270.77 & 78.29 \\ \hline
\bf VPMN & 188.09 & 193.28 & 96.17 \\ \hline
\end{tabular}
\end{center}
\begin{center}
\caption{Taxa màxima de transferència}
\label{T:tax}
\end{center}
\end{table}
L'aplicació que arriba a la màxima taxa és la OpenVPN amb tràfic UDP, és el valor més proper a la connexió directe només superant lleugerament la meitat d'aquest últim valor.
L'aplicació TincVPN té valors molt baixos en TCP.
El programa creat en aquest projecte (VPMN), malgrat tenir un bon resultat en TCP i MGEN, es manté en velocitats semblants tant en TCP com en UDP.

Per tant es pot observar com en treballar amb paquets grans els resultats són molt diferents dels d'eficiència. En aquest test però també s'hi veu representat l'eficiència dels algoritmes interns, de xifratge i d'enrutament (OpenVPN només enruta en un node).

El test del MGEN al utilitzar UDP hauria de donar resultats semblants als del test de UDP però en canvi dona resultats ben diferents. Primerament no arriba a enviar els aproximadament 740 Mbps, sinó uns 116 Mbps. Per tant el programari està treballant en condicions més normals de les de taxa màxima. Concloent, el resultat del test amb MGEN dona uns resultats estranys possiblement per la manera de mesurar la velocitat.

A continuació com es fa un \textbf{resum} de la secció.
El OpenVPN realitza una inicialització extensa, te una latència gran juntament amb una desviació estàndard alta i permet taxes de transferència grans.
El TincVPN realitza una inicialització molt breu, te una latència molt baixa i permet taxes de transferència baixes.
L'aplicació creada realitza una inicialització breu, te una latència baixa juntament amb una desviació estàndard molt baixa i permet taxes de transferència altes en TCP i baixes en UDP (en comparació amb la de les altres aplicacions).

\chapter{Conclusions}
En el projecte s'ha dedicat molt de temps en trobar un protocol de seguretat tolerant a pèrdues, i per tant un protocol que permeti esquivar el problema ja esmentat de TCP dins de TCP. En trobar DTLS i la seva implementació en OpenSSL es va procedir a dissenyar i implementar l'aplicació.

Més endavant el projecte es va veure endarrerit pel problema de la recent implementació de DTLS d'aquesta llibreria.
Tal com s'ha comentat, el poc ús que rep aquest protocol i l'implementació de OpenSSL, va fer que fos més difícil trobar documentat l'error. Això també ha fet que la dedicació del projecte es centres més a solucionar aquest error, tant dins de la aplicació com en la llibreria, que en implementar totes les altres funcionalitats.

Per tant funcionalitats com la del temporitzador (explicat en la secció \ref{arch-temp}) o un interpret pel fitxer de configuració no han estat implementades, tot i que tota la estructura està dissenyada i preparada per que hi acabin existint.

De les diferents eleccions que s'han pres durant el disseny del projecte es pot dir que l'elecció del protocol DTLS no ha estat un error, tal com s'ha vist amb els resultats dels tests en l'anterior capítol. Tampoc és una mala decisió utilitzar el OpenSSL al ser la única implementació de DTLS trobada. En canvi la elecció de realitzar el projecte amb \emph{threads} ha comportat una dedicació molt més gran de la que s'hagi pogut apreciar en els resultats; però tal com s'ha comentat, faltaria fer tests amb més nodes per veure el possible benefici d'aquesta decisió.

En un futur es pretén deixar l'aplicació operativa i oberta a la comunitat de programari lliure.
Per tant arreglar l'error del codi font de la llibreria OpenSSL és una prioritat a curt termini, així com també fer més usable l'aplicació per l'usuari final.

Com a curiositat aquest projecte va ser presentat \emph{Summer Camp Garrotxa 2008} el dia 5 de Juliol del 2008 a la població de Sant Jaume de Llierca, la presentació va atraure al públic present deixant els interessats a les expectatives d'una aplicació totalment funcional.

% Questions
\begin{frame}
  \vbox{}
  \vfill
  \begin{centering}
    \begin{beamercolorbox}[sep=8pt,center,colsep=-4bp,rounded=true]{title}
      \usebeamerfont{title}\inserttitle\par%
      \ifx\insertsubtitle\@empty%
      \else%
        \vskip0.25em%
        {\usebeamerfont{subtitle}\usebeamercolor[fg]{subtitle}\insertsubtitle\par}%
      \fi%     
    \end{beamercolorbox}%
    \vskip1em\par
    \begin{beamercolorbox}[sep=8pt,center,colsep=-4bp,rounded=true]{author}
      \usebeamerfont{author}\insertauthor
    \end{beamercolorbox}
    \begin{beamercolorbox}[sep=8pt,center,colsep=-4bp,rounded=true]{institute}
      \usebeamerfont{institute}
      \textit{Enginyeria de Telecomunicaci\'o (segon cicle)}\\~\\
      \usebeamerfont{institute}\insertinstitute
    \end{beamercolorbox}
  \end{centering}
  \vfill
\end{frame}
\end{document}
