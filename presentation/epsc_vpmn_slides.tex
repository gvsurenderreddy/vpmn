\documentclass{beamer}
\mode<presentation>
{
  %\usetheme{Warsaw}
  %\setbeamercovered{transparent}
  % or ...
  \usetheme{Darmstadt}
  \usefonttheme[onlylarge]{structurebold}
  \setbeamerfont*{frametitle}{size=\normalsize,series=\bfseries}
  %\setbeamertemplate{navigation symbols}{}
}

\usepackage[english]{babel}
% or whatever

\usepackage[utf8]{inputenc}
% or whatever

\usepackage{times}
\usepackage[T1]{fontenc}
% Or whatever. Note that the encoding and the font should match. If T1
% does not look nice, try deleting the line with the fontenc.


\title[VPMN]{Virtual Private Mesh Network} % (optional, use only with long paper titles)

\subtitle{Decentralized VPN Application} % (optional)

\author[Pau]{Pau~Rodriguez-Estivill}

\institute[EPSC UPC] % (optional, but mostly needed)
{
  Escola Polit\`ecnica Superior de Castelldefels\\
  Universitat Polit\`ecnica de Catalunya
}
% - Use the \inst command only if there are several affiliations.
% - Keep it simple, no one is interested in your street address.

\date[date] % (optional)
{\today}

\subject{Projecte de Fi de Carrera}
% This is only inserted into the PDF information catalog. Can be left
% out. 

% If you have a file called "university-logo-filename.xxx", where xxx
% is a graphic format that can be processed by latex or pdflatex,
% resp., then you can add a logo as follows:

% \pgfdeclareimage[height=0.5cm]{university-logo}{university-logo-filename}
% \logo{\pgfuseimage{university-logo}}

\pgfdeclareimage[height=0.5cm]{epsc-logo}{../epsc/logo_epsc}
\logo{\pgfuseimage{epsc-logo}}


\AtBeginSection[]
{
  \begin{frame}<beamer>{Outline}
    \tableofcontents[currentsection]
  \end{frame}
}

\AtBeginSubsection[]
{
  \begin{frame}<beamer>{Outline}
    \tableofcontents[currentsection,currentsubsection]
  \end{frame}
}

% If you wish to uncover everything in a step-wise fashion, uncomment
% the following command: 

%\beamerdefaultoverlayspecification{<+->}

\newcommand*{\comparativeframe}[5]{
\begin{frame}{#1}
	#2
	\begin{exampleblock}{Pros}
		\begin{itemize}
		#3
		\end{itemize}
	\end{exampleblock}
	\begin{alertblock}{Cons}
		\begin{itemize}
		#4
		\end{itemize}
	\end{alertblock}
	#5
\end{frame}
}

\begin{document}
\pdfbookmark[2]{Welcome}{welcome}
\begin{frame}
  \titlepage
\end{frame}

\begin{frame}{Outline}
  \tableofcontents
  % You might wish to add the option [pausesections]
\end{frame}

% However, the talk length of between 15min and 45min and
% the theme suggest that you stick to the following rules:

% - Exactly two or three sections (other than the summary).
% - At *most* three subsections per section.
% - Talk about 30s to 2min per frame. So there should be between about
%   15 and 30 frames, all told.

\section{Introduction}
\subsection*{Status}
\begin{frame}{Internet}
	\begin{itemize}
	\item \textbf{Insecure}
		\begin{itemize}
		\item Traffic can be read by other parties
		\item Traffic can be modified by other parties
		\item Content can be faked (\textbf{phishing})
		\end{itemize}
	\item \textbf{Untrusted}
		\begin{itemize}
		\item IPs can be spoofed
		\end{itemize}
	\item \textbf{Divided}
		\begin{itemize}
		\item NATs separate it in multiple private networks
		\end{itemize}
	\end{itemize}
\end{frame}
\begin{frame}{VPN}
	\begin{center}
	\includegraphics[height=7em]{images/vpn}
	\end{center}
	\begin{itemize}
	\item \textbf{Virtual Network}
		\begin{itemize}
		\item a tunnel with no routing hops
		\item own IP addressing
		\end{itemize}
	\item \textbf{Private}
		\begin{itemize}
		\item exclusive for trusted parties
		\item traffic cannot be read by other parties
		\end{itemize}
	\end{itemize}
\end{frame}
\begin{frame}{Security}
	\begin{itemize}
	\item \textbf{Encrypting} it cannot be read by other parties
	\item \textbf{Integrity validation} ensure it has not been modified
	\item \textbf{Authenticating} ensure that it is from a trusted party
	\item \textbf{No repudiation} other parties cannot lie
	\item \textbf{Anti-replay} protect against malicious replay
	\end{itemize}
\end{frame}
\subsection*{Objectives}
\begin{frame}{Project Aims}
	\begin{center}
	\includegraphics[height=10em]{images/vpn-fullymeshed}
	\end{center}
	\begin{itemize}
	\item Add dynamically nodes to the fully connected mesh
	\item Authenticate nodes and IP addressing together
	\item Low latency and low overhead
	\item NAT friendly
	\end{itemize}
\end{frame}

\section{State of Art}
\subsection{IPsec}
\comparativeframe{IPsec vs VPMN}{}
{% Pros
\item Standard
\item Mandatory in IPv6 implementations
\item DNSSEC enable possible dynamic tunneling
}{% Cons
\item Different implementations are \alert{not compatible}
\item Must be supported in kernel
\item Only one mode supported through NAT
\item IP addressing authentication not centralized
}{}
\subsection{OpenVPN}
\comparativeframe{OpenVPN vs VPMN}
{\begin{center}
\includegraphics[height=6em]{images/vpn-centralized}
\end{center}}
{% Pros
\item IP configurations can be pushed
\item Standard encryption channel
}{% Cons
\item Centralized, mesh alternative not easy
\item IP addressing not authenticated
}{}
\subsection{Tinc VPN}
\comparativeframe{Tinc VPN vs VPMN}
{\begin{center}
\includegraphics[height=6em]{images/vpn-meshed}
\end{center}}
{% Pros
\item Meshed Network
}{% Cons
\item Not fully connected
\item IP addressing not authenticated
}{}

\section{Internals}
\subsection{Overview}
\begin{frame}{Application}
	\begin{center}
	\includegraphics[height=10em]{images/dia-app}
	\end{center}
	\begin{itemize}
	\item Multi-thread application 
	\item Entirely written in C
	\end{itemize}
\end{frame}
\subsection{Security}
\begin{frame}{Security summary}
	\begin{itemize}
	\item Datagram TLS (DTLS)
	\item Uses CA for trusting nodes from the same network
	\item Certificates contain ACLs for IP addressing
	\begin{block}{NameConstraints (x509v3)}
	nameConstraints=permitted;IP:192.168.0.0/255.255.0.0
	\end{block}
	\end{itemize}
\end{frame}
\begin{frame}{DTLS: Datagram TLS}
	\begin{itemize}
	\item\begin{thebibliography}\beamertemplatearticlebibitems
	\item RFC 4347
	\end{thebibliography}
	\item All operations in UDP
		\begin{itemize}
		\item Exchange of certificates
		\item Negotiation of cryptography algorithms
		\item Transport of ciphered data
		\end{itemize}
	\item Based on TLSv1
	\item Cryptography mechanisms as TLS 
		\begin{itemize}
		\item Encryption
		\item Integrity validation
		\item Authentication
		\item Anti-replay mechanism
		\end{itemize}
	\end{itemize}
\end{frame}
\begin{frame}{Type of packets}
	\begin{enumerate}
	\item Raw IP packets
	\vspace{1em}
	\item Identification packets
		\begin{itemize}
		\item IP-port pairs
		\item Shared networks
		\end{itemize}
	\item Identification acknowledgment packets
	\item Keep alive packets
		\begin{itemize}
		\item Information of other known peers
			\begin{itemize}
			\item All IP-port pairs
			\item Shared networks
			\end{itemize}
		\end{itemize}
	\end{enumerate}
\end{frame}
\subsection{Architecture}
\begin{frame}{UDP Server Part}
	\begin{center}
	\includegraphics[height=18em]{images/dia-udpsrv}
	\end{center}
\end{frame}
\begin{frame}{TUN Server Part}
	\begin{center}
	\includegraphics[height=18em]{images/dia-tunsrv}
	\end{center}
\end{frame}
\subsection{Conclusions}
\comparativeframe{Conclusions}{}
{% Pros
\item Dynamic Fully Connected Mesh Network
\item Authenticated IP addressing with certificates
\item Standard encryption channel (\textbf{DTLS})
}{% Cons
\item No relay mode possible
\item Fragmentation needed
}{
\begin{block}{Future tasks}\begin{itemize}
\item Heterogeneous MTU support
\item Let fragmentation be optional
\item User-space NAT
\end{itemize}\end{block}
}
%\section{Tests}

\section*{Summary}
%\subsection*{Summary}
\begin{frame}{Summary}
	\begin{itemize}
	\item Internet is \alert{insecure} and \textbf{VPN} is needed
	\item \textbf{Server-less} architecture
	\item \textbf{Dynamic} Fully Connected Mesh Network
	\item \textbf{Authenticated IP} addressing with certificates
	\item Standard encryption channel (\textbf{DTLS})
	\end{itemize}
\end{frame}

% Questions
\begin{frame}
\begin{center}
\usebeamerfont{title}Questions?
\end{center}
\end{frame}
\end{document}
