\chapter{Programació}
\section{Funcionament}
L'aplicació en arrancar es posa en contacte amb els nodes que coneix. Per cada node s'intercanvien l'informació necessària per tal de poder identificar-se i enrutar els paquets.
\section{Disseny del Protocol}
\subsection{Seguretat}
L'objectiu es que tots els nodes puguin realitzar els túnels amb les garanties de les operacions criptogràfiques com el xifrat, la firma digital i el no repudi. De manera que cada node té les garanties de estar entregant l'informació directament al destí correcte i sap que cap altre pot estar ni llegint ni manipulant.

Per assolir aquest objectiu s'han de realitzar connexions directes entre tots els nodes, i no es pot realitzar agregacions de nodes com típicament es fa en xarxes \keyword{P2P}{Peer to peer} mitjançant nodes \emph{relay}. També es necessari un mecanisme de validació de les rutes que cada node ofereix, de manera que cap node no pugui tenir una direcció IP que pertanyi a un altre node. Aquesta última necessitat obliga a fixar un adreçament IP estàtic.

Per tant el primer que cal establir es el funcionament del adreçament IP, per a fer-ho es contemplen dos possibilitats:
\begin{itemize}
\item L'utilització de un adreçament que depengui dels identificadors dels certificats
\item L'incrustació de l'adreçament dins dels camps dels certificats
\end{itemize}
La primera opció consisteix en definir una distribució de direccions IP en la que a partir de un certificat se li pugui assignar una única IP.

\subsubsection{DTLS}

\subsection{Definició de paquets}
Per el disseny del protocol es necessita un identificador únic per cada node. Es va pensar i valorar per a aquesta aplicació les següents opcions:
\begin{itemize}
\item La IP i port del node
\item La IP publica i port del node
\item Dupleta IP local i publica
\item Xarxes que publicita el node
\end{itemize}
L'utilització de la IP i el port no és informació suficient degut a l'existència de xarxes privades i publiques, també els NAT simètrics no donen el mateix port per a tots els hosts remots on es connecti un mateix node.
L'utilització de la IP publica per aquesta tasca no és valida ja que deixa de ser única si hi ha dos nodes al costat privat abans de un NAT. Aquest dos en trobar-se es pensaran que són nodes diferents dels que coneixen amb IP pública.

\begin{quote}
Fragmentació a quin nivell.
Hole punching.
\end{quote}

\section{Arquitectura del programa}
\begin{quote}
Llenguatge C.
Threads.
\end{quote}
La arquitectura del programa es separa entre l'adaptador virtual i el servidor UDP. Els paquets que les aplicacions envien a traves de l'adaptador virtual son enrutats i enviats de servidor UDP a servidor UDP on retornen a un altre adaptador virtual a la màquina de destí.
Pel que fa al adaptador virtual s'utilitza el controlador universal TUN/TAP.
\subsection{Interfície TUN}

\subsection{Servidor UDP/DTLS}
