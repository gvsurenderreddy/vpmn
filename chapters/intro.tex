Internet actualment depèn extensivament de \keyword{IP}{Internet Protocol (versió 4 o 6)}v4, i veient que a IPv6 encara li queda camí per assolir la implementació global, es pot dir que la limitació d'adreçament públic que provoca IPv4 seguirà existint a mig termini. La funció de permetre la unió de diferents ordenadors o \keyword{LAN}{Local Area Network}s en una nova xarxa virtual, pot ser aprofitada per poder abstraure's de les barreres imposades per IPv4. La creació d'aquestes xarxes virtuals és molt útil, però s'ha de tenir en compte que qualsevol que hi tingui accés serà com si estigués connectat directament amb les LANs i ordenadors que en formen part. És per això que casi tot programari que ofereix la creació d'aquestes xarxes virtuals, també ofereix la possibilitat d'afegir una capa de seguretat, que pot incloure tant l'autenticació com l'encriptació del canal de dades. Aquesta solució es denomina \keyword{VPN}{Virtual Private Network} i és la més utilitzada en tots els àmbits (de fet casi tota xarxa virtual utilitzada és una VPN).
També és una realitat que la majoria d'aplicacions ja disposen d'una capa de seguretat pel transport de les seves pròpies dades, però la utilització de VPNs permet forçar un entorn segur, de confiança i independent de la de les aplicacions. 

L'objectiu d'aquest projecte és dissenyar i implementar una aplicació capaç de crear xarxes privades virtuals que no depenguin de cap servidor central, sense que això comprometi la privacitat ni l'autenticació dels integrants de la xarxa. L'aplicació ha de ser capaç de superar els routers \keyword{NAT}{Network Address Translator} per tal d'establir connexions bidireccionals amb els veïns de la xarxa, proporcionant una baixa latència. També és interessant la utilització d'adreçament autenticat, de manera que ningú no pugui falcejar la identitat dins de la VPN.

%Resum dels capítols
Aquest document comença amb la descripció i la comparativa de les tecnologies de VPN existents.
Desprès s'explica el funcionament, el disseny i l'arquitectura de l'aplicació creada.
Seguidament es presenten els resultats de les proves realitzades amb l'aplicació creada. 
I finalment hi ha les conclusions, la bibliografia i el glossari.
