Internet actualment depèn extensivament de \index{IP}IP\footnote{Internet Protocol (versió 4 o 6)}v4 i veient que a la IPv6 encara li queda camí per convèncer per la implementació global. La funció de permetre la unió de diferents ordenadors o \index{LAN}LAN\footnote{Local Area Network}s en una nova xarxa virtual, pot ser aprofitada per poder abstraure's de les barreres que imposa la limitació d'adreçament públic de IPv4. La creació d'aquestes xarxes virtuals és molt útil, però s'ha de tenir en compte que qualsevol que tingui accés a aquesta xarxa serà com si estigués connectat directament amb les LANs i ordenadors que en formen part. És per això que casi tot programari que ofereix la creació d'aquestes xarxes virtuals, també ofereix la possibilitat d'afegir una capa de seguretat, que inclou la autenticació i també pot incloure l'encriptació del canal de dades. Aquesta solució es denomina \index{VPN}VPN\footnote{Virtual Private Network} i es la més utilitzada en tots els àmbits, casi tota xarxa virtual utilitzada és una VPN.
També es una realitat que la majoria d'aplicacions ja disposen d'una capa de seguretat pel transport de les pròpies dades, que faria poc útil haver de tornar a encriptar al passar per les VPNs.

La idea és la de desenvolupar una aplicació capaç de crear xarxes completament mallades a nivell de aplicació.
