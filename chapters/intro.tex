Internet actualment depèn extensivament de \keyword{IP}{Internet Protocol (versió 4 o 6)}v4, i veient que a la IPv6 encara li queda camí per assolir la implementació global, es pot dir que la limitació d'adreçament públic que provoca IPv4 seguirà existint fins a un futur no molt proper. La funció de permetre la unió de diferents ordenadors o \keyword{LAN}{Local Area Network}s en una nova xarxa virtual, pot ser aprofitada per poder abstraure's de les barreres que imposades per IPv4. La creació d'aquestes xarxes virtuals és molt útil, però s'ha de tenir en compte que qualsevol que hi tingui accés serà com si estigués connectat directament amb les LANs i ordenadors que en formen part. És per això que casi tot programari que ofereix la creació d'aquestes xarxes virtuals, també ofereix la possibilitat d'afegir una capa de seguretat, que pot incloure tant l'autenticació com l'encriptació del canal de dades. Aquesta solució es denomina \keyword{VPN}{Virtual Private Network} i és la més utilitzada en tots els àmbits (de fet casi tota xarxa virtual utilitzada és una VPN).
També és una realitat que la majoria d'aplicacions ja disposen d'una capa de seguretat pel transport de les pròpies dades, però l'utilització de VPNs permet forçar un entorn segur de confiança independent de la de les aplicacions. 

L'objectiu d'aquest projecte és el de crear una aplicació capaç de crear xarxes privades virtuals que no depenguin de cap servidor central, sense que això comprometi la privacitat ni l'autenticació dels integrants de la xarxa. La idea és que l'aplicació sigui capaç de superar els routers \keyword{NAT}{Network Address Translator} per tal d'establir connexions bidireccionals amb els veïns de la xarxa, proporcionant una baixa latència. També és interessant l'utilització d'adreçament autenticat, de manera que ningú no pugui falcejar l'identitat dins de la VPN.

%TODO: Resum dels capítols
