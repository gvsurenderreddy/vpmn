\chapter{Tests i resultats}
\section{Tests}
En aquest capítol s'exposen els resultats dels tests realitzats per comparar l'aplicació creada amb el programari existent.
En tots els programes s'han utilitzat certificats de 1024 bits i s'ha activat la compressió que permetés l'aplicació; el motiu d'activar la compressió és que la llibreria OpenSSL utilitzada per la creació d'aquest programa no permet desactivar la compressió en temps d'execució.
L'escenari dels tests consta de una xarxa Gigabit Ethernet on hi estan connectats dos ordenadors:
\begin{enumerate}
\item Debian GNU/Linux (PC amd64)
\begin{itemize}
\item AMD Athlon 64 X2 Dual Core Processor BE-2400 (2.3GHz, 1024KB cache)
\item nForce Gigabit Ethernet CK804
\end{itemize}
\item Debian GNU/Linux (PC i686)
\begin{itemize}
\item Intel Pentium M Centrino Processor (1.70GHz, 2048KB cache)
\item Realtek Gigabit Ethernet RTL-8169
\end{itemize}
\end{enumerate}
%TODO:
%Afegir retards als tests
%Tests de Compressio
\subsection{Eficiència}
\begin{table}[htb]
\begin{center}
\begin{tabular}{|c|c|c|}
\multicolumn{1}{c}{} & \multicolumn{2}{|c|}{Mida (bytes)} \\ \hline
Aplicació & Inicialització & Ping \\ \hline \hline
\tt directe & ARP? & 84 \\ \hline
OpenVPN & 0 & 0 \\ \hline
TincVPN & 0 & 0 \\ \hline
\bf VPMN & 0 & 0 \\ \hline
\end{tabular}
\end{center}
\begin{center}
\caption{Eficiència}
\label{T:efi}
\end{center}
\end{table}

\subsection{Latència}
\begin{table}[htb]
\begin{center}
\begin{tabular}{|c|c|c|c|c|c|}
\multicolumn{2}{c}{} & \multicolumn{4}{|c|}{RTT (ms)} \\ \hline
Aplicació & Pèrdues & Mínim & Mitjana & Màxim & D.Estàndard \\ \hline \hline
\tt directe & 0\% & 0 & 0 & 0 & 0 \\ \hline
OpenVPN & 0\% & 0 & 0 & 0 & 0 \\ \hline
TincVPN & 0\% & 0 & 0 & 0 & 0 \\ \hline
\bf VPMN & 0\% & 0 & 0 & 0 & 0 \\ \hline
\end{tabular}
\end{center}
\begin{center}
\caption{Latència}
\label{T:lat}
\end{center}
\end{table}

\subsection{Taxa màxima de tranferencia}
\begin{table}[htb]
\begin{center}
\begin{tabular}{|c|c|}
\hline
Aplicació & Màxim (Mbps) \\ \hline \hline
\tt directe & 0 \\ \hline
OpenVPN & 0 \\ \hline
TincVPN & 0 \\ \hline
%\bf VPMN & 0 \\ \hline
\end{tabular}
\end{center}
\begin{center}
\caption{Taxa màxima de transferencia}
\label{T:tax}
\end{center}
\end{table}

%Proves amb varios nodes
%\subsection{Capacitat}
%\subsection{Càrrega}
%\subsection{Escalabilitat}
\section{Anàlisi de resultats}
